\documentclass{VUMIFPSkursinis}
\usepackage{algorithmicx}
\usepackage{algorithm}
\usepackage{algpseudocode}
\usepackage{amsfonts}
\usepackage{amsmath}
\usepackage{bm}
\usepackage{caption}
\usepackage{color}
\usepackage{float}
\usepackage{graphicx}
\usepackage{listings}
\usepackage{subfig}
\usepackage{wrapfig}
\usepackage{enumitem}
\usepackage[utf8]{inputenc}
%\usepackage{enumerate}

%PAKEISTA, tarpai tarp sąrašo elementų
\setitemize{noitemsep,topsep=0pt,parsep=0pt,partopsep=0pt}
\setenumerate{noitemsep,topsep=0pt,parsep=0pt,partopsep=0pt}

% Titulinio aprašas
\university{Vilniaus universitetas}
\faculty{Matematikos ir informatikos fakultetas}
\department{Programų sistemų katedra}
\papertype{Laboratorinis darbas}
\title{Kelionių į mėnulį maršrutų programa „Poon“}
%\titleineng{A project for a music hoarding program MusiX}
\status{3 kurso 5 grupės studentai}
\author{Gabrielė Žielytė, Daumantas Šimkus}
\secondauthor {Nedas Valentinovičius}   % Pridėti antrą autorių 
\titleineng{}

\supervisor{dr. Vytautas Čyras}
\date{Vilnius – \the\year}

% Nustatymai
\setmainfont{Palemonas}   % Pakeisti teksto šriftą į Palemonas (turi būti įdiegtas sistemoje)
\bibliography{bibliografija}

\begin{document}
	
% PAKEISTA	
\maketitle
\cleardoublepage\pagenumbering{arabic}
\setcounter{page}{2}

%TURINYS
\tableofcontents

\sectionnonum{Anotacija}
Šiame dokumente aprašoma bilietų pirkimo kelionėms į mėnulį telefoninė programa „Poon“. Darbo tikslas - sukurti patogią, vartotojui suprantamą ir intuityvią vartotojo sąsają.
Studentų, dirbusių prie šio projekto, kontaktai bei indėlis:
\begin{itemize}
\item Gabrielė Žielytė - email
\item Daumantas Šimkus - email
\item Nedas Valentinovičius - email
\end{itemize}

\section{Įvadas}
Programų sistemos ilgasis pavadinimas yra „Kelionių į Mėnulį maršrutų planavimo programa“, o trumpasis pavadinimas - „Poon“

Ši programa spręs šiuo metu žmonėms dažnai iškylančias problemas, susijusias su skrydžiais į Mėnulį. Egzistuoja daugybė programų, galinčių pateikti skrydžių maršrutus ar leidžiančių užsisakyti bilietus. Bėda tame, kad nera tokios sistemos, kuri leistų atlikti visus su bilieto pirkimu susijusius žingsnius - tokius kaip tvarkaraščio peržiūra, maršrutų matymas, vieno ar daugiau bilietų užsisakymas. Poon tikslas - supaprastinti visą pirkimo procesą nuo tvarkaraščių peržiūros iki pat gryžimo atgal į Žemę. 

\section{Suinteresuotieji}
Apie suinteresuotuosiuos (keleivius ir apie dar ka nors, kam idomi sita programele)

\section{Vartotojų poreikia}
Šiame skyriuje bus pateikiami įvairių vartotojų rušių poreikiai bei būsimieji veiksmų scenarijai.
\subsection{Keleivių poreikiai}
Kadangi keleiviai sudarys didžiąją dalį vartotojų, šios programos vartotojo sąsaja turi buti pritaikyta butent jiems - greita bei aiški pažiūrejus vos kelias akimirkas.
\subsubsection{Einamųjų veiklų analizė}
fuck knows
\subsubsection{Naudotojų ir veiklų charakteristikų analizė}
fucker knows
\subsubsection{Būsimasis naudojimo scenarijus}
Vartotojas nori užsisakyti skrydį į Mėnulį, kuris iš Žemės išvyktų vakare.
\begin{enumerate}
\item Įveda pageidaujamą išvykimo iš Žemės laiką.
\begin{enumerate}[label*=\arabic*.]
\item Pasirenka, kad skris tik į vieną pusę.
\end{enumerate}
\item Sistema pateikia sąrašą skrydžių, kurie vyksta iš Žemės ir kurių išvykimo laikas yra ne ankstesnis nei nurodyta vartotojo. Taip pat yra nurodyta skrydžio kaina.
\item Pasirinkęs norimą skrydį vartotojas mato išvykimo ir atvykimo datas, skrydžio kainą, planuojamą laiką bei galimas sėdimas vietas.
\item Vartotojas suveda asmens dokumento duomenis, elektroninio pašto adresą.
\item Pasirenka apmokėjimo būdą, bei susimoka už kelionę.
\item Sumokėjus vartotojas gauna bilietus į savo el.paštą.
\end{enumerate}
\subsubsection{Naudotojų poreikiai ir panaudojamumo siekiai}
net nezinau, ar sito reikia lol


\printbibliography[heading=bibintoc, title=Šaltiniai]  % Šaltinių sąraše nurodoma panaudota
\end{document}
