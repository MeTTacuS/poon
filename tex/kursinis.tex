\documentclass{VUMIFPSkursinis}
\usepackage{algorithmicx}
\usepackage{algorithm}
\usepackage{algpseudocode}
\usepackage{amsfonts}
\usepackage{amsmath}
\usepackage{bm}
\usepackage{caption}
\usepackage{color}
\usepackage{float}
\usepackage{graphicx}
\usepackage{listings}
\usepackage{subfig}
\usepackage{wrapfig}
\usepackage{enumitem}
\usepackage[utf8]{inputenc}
%\usepackage{enumerate}

%PAKEISTA, tarpai tarp sąrašo elementų
\setitemize{noitemsep,topsep=0pt,parsep=0pt,partopsep=0pt}
\setenumerate{noitemsep,topsep=0pt,parsep=0pt,partopsep=0pt}

% Titulinio aprašas
\university{Vilniaus universitetas}
\faculty{Matematikos ir informatikos fakultetas}
\department{Programų sistemų katedra}
\papertype{Laboratorinis darbas}
\title{Kelionių į mėnulį maršrutų programa „Poon“}
%\titleineng{A project for a music hoarding program MusiX}
\status{3 kurso 5 grupės studentai}
\author{Gabrielė Žielytė, Daumantas Šimkus}
\secondauthor {Nedas Valentinovičius}   % Pridėti antrą autorių 
\titleineng{}

\supervisor{dr. Vytautas Čyras}
\date{Vilnius – \the\year}

% Nustatymai
\setmainfont{Palemonas}   % Pakeisti teksto šriftą į Palemonas (turi būti įdiegtas sistemoje)
\bibliography{bibliografija}

\begin{document}
	
% PAKEISTA	
\maketitle
\cleardoublepage\pagenumbering{arabic}
\setcounter{page}{2}

%TURINYS
\tableofcontents

\sectionnonum{Anotacija}
Šiame dokumente aprašoma bilietų pirkimo kelionėms į mėnulį telefoninė programa „Poon“. Darbo tikslas - sukurti patogią, vartotojui suprantamą ir intuityvią vartotojo sąsają.
Studentų, dirbusių prie šio projekto, kontaktai bei indėlis:
\begin{itemize}
\item Gabrielė Žielytė - email
\item Daumantas Šimkus - email
\item Nedas Valentinovičius - email
\end{itemize}

\section{Įvadas}
Programų sistemos ilgasis pavadinimas yra „Kelionių į Mėnulį maršrutų planavimo programa“, o trumpasis pavadinimas - „Poon“

Ši programa spręs šiuo metu žmonėms dažnai iškylančias problemas, susijusias su skrydžiais į Mėnulį. Egzistuoja daugybė programų, galinčių pateikti skrydžių maršrutus ar leidžiančių užsisakyti bilietus. Bėda tame, kad nėra tokios sistemos, kuri leistų atlikti visus su bilieto pirkimu susijusius žingsnius - tokius kaip tvarkaraščio peržiūra, maršrutų matymas, vieno ar daugiau bilietų užsisakymas. Poon tikslas - supaprastinti visą pirkimo procesą nuo tvarkaraščių peržiūros iki pat grįžimo atgal į Žemę.

Kelionėms į mėnulį tapus realybe, tapo ypač svarbu tvarkinga, aiški, moderni ir svarbiausia saugi kelionių planavimo sistema. Į mėnulį plūsta žmonės iš skirtingų kultūrų, tikėjimo, puoselėjantys įvairias vertybes. Todėl Poon misija yra gerbti kiekvieną norintįjį skristi, parūpinti jam saugią ir malonią aplinką su kuo aiškesnėmis taisyklėmis kelionėms tarp Žemės ir Mėnulio.

\centerline{Poon - prisiliesk prie Mėnulio ir tu!}



\section{Suinteresuotieji}

Šią programa norėtų naudotis visi pilnamečiai asmenys, norintys planuoti keliones į mėnulį. Kadangi pasaulio valstybėse pilnametystė yra reglamentuota skirtingai. Poon pilnametis asmuo yra tas, kuris pagal savo valstybės įstatymus, pilnai atsako už save ir turi teisę pirkti. Jų esminiai lūkesčiai būtų aiškus ir greitas skrydžių pasirinkimas bei lengva apmokėjimo sistema.

\section{Vartotojų poreikia}
Šiame skyriuje bus pateikiami įvairių vartotojų rušių poreikiai bei būsimieji veiksmų scenarijai.

\centerline{***** PAPILDYTI *****}

\subsection{Keleivių poreikiai}
Kadangi keleiviai sudarys didžiąją dalį vartotojų, šios programos vartotojo sąsaja turi buti pritaikyta butent jiems - greita bei aiški pažiūrejus vos kelias akimirkas.
\centerline{***** PAPILDYTI *****}

\subsubsection{Einamųjų veiklų analizė}
\centerline{***** PAPILDYTI *****}

\subsubsection{Naudotojų ir veiklų charakteristikų analizė}
\centerline{***** PAPILDYTI *****}

\subsubsection{Būsimasis naudojimo scenarijus}
Vartotojas su draugu nori skristi į mėnulį šios dienos pabaigoje:
\begin{enumerate}
\item  Vartotojas supildo informaciją apie skrydį.
\begin{enumerate}[label*=\arabic*.]
\item Pasirenka, kad skris tik į vieną pusę.
\item Pažymi, jog skris tik į vieną pusę.
\item Pasirenka, jog skris iš Žemės.
\item  Pasirenka, jog išvykti nori šiandien po 18:00 val.
\end{enumerate}
\item  Sistema pateikia sąrašą skrydžių, kurie vyksta iš Žemės ir kurių išvykimo laikas yra ne ankstesnis nei nurodyta vartotojo ir turi bent dvi laisvas vietas. Prie kiekvieno skrydžio yra nurodyta kaina.
\item Vartotojas pasirenką norimą skrydį.
\item Sistema parodo to skrydžio išvykimo ir atvykimo datas su planuojamais laikais, skrydžio kainą bei galimas sėdimas vietas.
\item Vartotojas pasirenka sėdimas vietas ir spaudžia mokėti.
\begin{enumerate}[label*=\arabic*.]
\item Vartotojas suveda asmens dokumento duomenis.
\item Įveda elektroninio pašto adresą.
\item Pasirenka apmokėjimo būdą.
\item  Susimoka už kelionę.
\end{enumerate}
\item   Sumokėjus sistema išsiunčia  bilietus į el.paštą.
\end{enumerate}

\subsubsection{Naudotojų poreikiai ir panaudojamumo siekiai}
\centerline{***** PAPILDYTI jei reikia*****}


\printbibliography[heading=bibintoc, title=Šaltiniai]  % Šaltinių sąraše nurodoma panaudota
\end{document}
