\documentclass{VUMIFPSkursinis}
\usepackage{algorithmicx}
\usepackage{algorithm}
\usepackage{algpseudocode}
\usepackage{amsfonts}
\usepackage{amsmath}
\usepackage{bm}
\usepackage{caption}
\usepackage{color}
\usepackage{float}
\usepackage{graphicx}
\usepackage{listings}
\usepackage{subfig}
\usepackage{wrapfig}
\usepackage{enumitem}
\usepackage[utf8]{inputenc}
%\usepackage{enumerate}

%PAKEISTA, tarpai tarp sąrašo elementų
\setitemize{noitemsep,topsep=0pt,parsep=0pt,partopsep=0pt}
\setenumerate{noitemsep,topsep=0pt,parsep=0pt,partopsep=0pt}

% Titulinio aprašas
\university{Vilniaus universitetas}
\faculty{Matematikos ir informatikos fakultetas}
\department{Programų sistemos}
\papertype{Laboratorinis darbas}
\title{Kelionių į Mėnulį maršrutų planavimo programa „Poon“}
\status{3 kurso 5 grupės studentai}
\author{Gabrielė Žielytė}
\secondauthor {Daumantas Šimkus}
\thirdauthor {Nedas Valentinovičius}
\titleineng{}

\date{Vilnius – \the\year, Galutinė versija}

% Nustatymai
\setmainfont{Palemonas}   % Pakeisti teksto šriftą į Palemonas (turi būti įdiegtas sistemoje)
\bibliography{bibliografija}

\begin{document}
	
% PAKEISTA	
\maketitle

%TURINYS
\thispagestyle{empty}
\tableofcontents


\sectionnonumnocontent{Anotacija}
Šiame dokumente aprašoma bilietų pirkimo kelionėms į mėnulį telefoninė programa „Poon“. Darbo tikslas - sukurti patogią, vartotojui suprantamą ir intuityvią vartotojo sąsają.
Studentų, dirbusių prie šio projekto, kontaktai bei indėlis:
\begin{itemize}
\item Gabrielė Žielytė - gabriele.zielyte@mif.stud.vu.lt
\item Daumantas Šimkus - daumantas.simkus@mif.stud.vu.lt
\item Nedas Valentinovičius - nedas.valentinovicius@mif.stud.vu.lt
\end{itemize}
\thispagestyle{empty}

\cleardoublepage\pagenumbering{arabic}
\setcounter{page}{4}

\section{Įvadas}
\textbf{Programų sistemos pavadinimas: } Kelionių į Mėnulį maršrutų planavimo programa „Poon“
\bigskip

\textbf{Trumpasis pavadinimas: } „Poon“
\bigskip




\textbf{Projekto aprašas: } Kelionėms į Mėnulį tapus realybe, tapo ypač svarbu tvarkinga, aiški, moderni ir svarbiausia saugi kelionių planavimo sistema. Į Mėnulį plūsta žmonės iš skirtingų kultūrų, tikėjimo, puoselėjantys įvairias vertybes. Todėl Poon misija yra gerbti kiekvieną norintįjį skristi, parūpinti jam saugią ir malonią aplinką su kuo aiškesnėmis taisyklėmis kelionėms tarp Žemės ir Mėnulio.

\bigskip
\textbf{Sprendžiamos problemos: }
\begin{enumerate}
\item Šiuo metu bilietai į Mėnulį perkami individualiai iš skrydžio kompanijų, o „Poon“ sistema jas visas apjungs.
\item Maršrutų patikrinimas ir bilietų pirkimas yra atskiri veiksmai, kuriuos ši programa apjungia į vieną vartotojo sąsają
\item Netikslūs, su skryžių kompanijomis nesusinchronizavę maršrutų tvarkaraščių puslapiai klaidina vartotojus, o „Poon“ programoje matomi tvarkaraščiai bus sinchronizuojami tiesiogiai su skrydžių bendrovėmis.
\item Šiuo metu kelių bilietų pirkimas yra itin komplikuotas, o „Poon“ programoje tai bus taip pat paprasta, kaip pirkti vieną bilietą.
\item Nėra galimybės iškart užsisakyti bilietą atgal, tačiau ši programa išsprendžia šią problemą iškart siūlydama įsigyti bilietus atgal su bet kuria atgal parskraidinti galinčia skrydžių kompanija.
\item Nėra patogios statistikos apie kiekvieną skrydžių kompaniją. „Poon“ anonimiškai ves statistiką ir ja dalinsis su visu pasauliu, kad būtų aiškiai matomi skirtingų skrydžių bendrovių populiarumai tarp vartotojų.
\end{enumerate}

\section{Suinteresuotieji}
Šio skyriaus tikslas yra išskirti suinteresuotųjų asmenų grupes tam, kad būtų galima nustatyti, kokie funkcionalumai yra reikalingi, kokie tiksliai yra naudotojų poreikiai, kam ši sistema bus naudingiausia bei atsižvelgti į konkurentų sistemas, jų neišnaudotas galimybes bei galimas tobulintinas sritis.

\subsection{Pirminiai} 
Šios programos pirminiai suinteresuotieji yra \textbf{pilnamečiai asmenys, norintys planuoti keliones į Mėnulį}. Kadangi pasaulio valstybėse pilnametystė yra reglamentuota skirtingai. Poon pilnametis asmuo yra tas, kuris pagal savo valstybės įstatymus, pilnai atsako už save. Pavyzdys būtų mokslinius tyrimus atliekantys asmenys. Kadangi skrydžiai į Mėnulį įprastu dalyku tapo neseniai, šis dangaus kūnas vis dar traukia mokslininkus bei tyrinėtojus. Programa, leidžianti užsisakyti bilietus tiek šiandien, tiek už mėnesio vykstantiems skrydžiams labai praverstų staiga į tyrimų stotį norintiems skristi ar po kurio laiko vykstančius tyrimus planuojantiems mokslininkams. 

Pirminių suinteresuotųjų esminiai lūkesčiai būtų:
\begin{enumerate}
\item Aiškus ir greitas skrydžių pasirinkimas
\item Lengva apmokėjimo sistema
\item Galimybė bilietus saugoti ilgą laiką
\item Gebėjimas bilietus perduoti kitiems asmenims
\end{enumerate}

\subsection{Antriniai} 
Antriniai šios programos suinteresuotieji yra \textbf{statistinių analizių vedėjai}. Ši suinteresuotųjų grupė naudosis mūsų programos pateikiamomis statistikomis tam, kad galėtų gauti tikslią bei visas skrydžių įmones apimančią statistiką. Šiai grupei svarbu žinoti, kiek mėnesinių klientų turi skirtingos skrydžių kompanijos, kurie skrydžiai populiariausi ir panašias statistikas, kurias „Poon“ programa ves.

Antrinių suinteresuotųjų esminiai lūkesčiai būtų:
\begin{enumerate}
\item Tiksli bei aiški statistinė skrydžų analizė
\item Duomenų anonimiškumas
\end{enumerate}

\subsection{Tretiniai} Tretiniai suinteresuotieji būtų šios \textbf{sistemos konkurentai}, nes jų veiklą veikia šios sistemos sėkmė arba nesėkmė. Šiuo metu rinkoje egzistuoja ne viena kompanija, organizuojanti skrydžius į Mėnulį. Kiekviena iš jų turi savą bilietų pirkimo sistemą, kur galima nusipirkti tik vienos skrydžių kompanijos bilietus, taip pat kiekviena naudojasi skirtinga mokėjimo sistema, o tai kelia vartotojams daug bėdų, ypač kai ilgą laiką buvo naudojamasi vienos kompanijos paslaugomis ir yra nuspręsta pradėti naudotis kitos kompanijos siūlomais skrydžiais. 

Konkretūs "Poon" konkurentų pavyzdžiai būtų "Comet express" ir "PlanetStar". Šios sistemos turi kelias neišnaudotas galimybės, kurias „Poon“ sieks pagerinti ir implementuoti į savąją sitema:
\begin{enumerate}
\item Konkurentų programos leidžia nusipirkti tik jų kompanijos bilietus, todėl bilietų ir datų skaičius yra gana ribotas
\item Jų vartotojo sąsajos atrodo ir yra naudojamos skirtingai, o „Poon“ sistemos sąsają bus galima naudoti prikti bilietus iš abiejų kompanijų
\item Programos savo statistikas viešina tik kartą per pusmetį, o „Poon“ sistemos teikiami duomenys bus matomi nuolatos.
\end{enumerate}

Poon bus primoji programa, leidžianti pirkti bilietus iš visų didžiųjų skrydžių organizatorių bei leidžianti susimokėti visais pagrindiniais mokėjimo būdais. Tai išspręs didžiąją dalį problemų, šiuo metu kylančių noritintiems skristi į Mėnulį asmenims, bei maždaug 15\% sumažins "Comet express" ir "PlanetStar" skrydžių planavimo sistemų vartotojų kiekį, nes didelė dalis vartotojų neabejotinai pradės naudoti daug patogesnę bei intuityvesnę „Poon“

\subsection{Kiti}
Kiti suinteresuotieji yra sistemos projektuotuojai, realizuotojai bei palaikytojai. Kiekviena žmonių grupė prie sistemos gerovės prisidės tokiais būdais:
\begin{enumerate}
\item Sistemos projektuotojai bei programuotojai Nedas Valentinovičius, Gabrielė Žielytė bei Daumantas Šimkus suprojektuos bei suprogramuos kitoms vartotojų grupėms patogią bei funkcionalią vartotojo sąsają.
\item Naudojamos duomenų bazės administratoriai užtikrins nuolatinį saugomų duomenų prieinamumą.
\item Statistinių duomenų centro darbuotojai užtikrins, kad „Poon“ vedami duomenys nėra klastojami bei tikslūs.
\end{enumerate}

\section{Keliautojų poreikiai}
Šiame skyriuje bus pateikiama detali keliautojų poreikių, siekių ir kitų tikslų analizė.

\subsection{Einamųjų veiklų analizė}
Toliau bus pateiktos kelios einamosios veiklos, analizuojamai naudotojų grupei keliančios nepatogumus.

\subsubsection{Pirma kompiuterizuojama veikla}
[TRUMPAS VIENO SAKINIO SCENARIJUAUS APRAŠYMAS]

\textit{TAS BLOGAS NAUDOJIMO SCENARIJUS, ŠIUO METU KELIANTIS ŽMOGUI PROBLEMAS}

Pateiktame apraše išdėstytos esamos veiklos problemos ir neišnaudotos galimybės:  
\begin{itemize}
\item ČIA SURAŠOMOS VISOS TOS PROBLEMOS, KURIOS MATOMOS SCENARIJUJE
\end{itemize}

\subsubsection{Antra kompiuterizuojama veikla}
[TRUMPAS VIENO SAKINIO SCENARIJUAUS APRAŠYMAS]

\textit{KITAS BLOGAS NAUDOJIMO SCENARIJUS, ŠIUO METU KELIANTIS ŽMOGUI PROBLEMAS}

Pateiktame apraše išdėstytos esamos veiklos problemos ir neišnaudotos galimybės:  
\begin{itemize}
\item ČIA SURAŠOMOS VISOS TOS PROBLEMOS, KURIOS MATOMOS SCENARIJUJE
\end{itemize}

\subsection{Naudotojų ir veiklų charakteristikos}
DAR VIENAS BLOKAS TEKSTO PAGAL TAI, KAS YRA PARAŠYTA DABARTINĖJE MŪSŲ BIBLIJOJE 3.4 SKYRIUJE

\subsection{Būsimieji panaudojimo scenarijai}
Toliau bus pateikti keli pavyzdiniai panaudojimo scenarijai, kurie padės išspręsti vartotojams kylančias problemas

\subsubsection{Pirmasis scenarijus}
VIENO SAKINIO APRAŠYMAS

KAIP KOMENTARUOSE SĄRAŠAS SU VISU SCENARIJUMI

\subsubsection{Antrasis scenarijus}
VIENO SAKINIO APRAŠYMAS

KAIP KOMENTARUOSE SĄRAŠAS SU VISU SCENARIJUMI

\subsection{Naudotojo poreikia bei panaudojamumo siekiai}
PARAŠYTI POREIKIUS, KURIE KILO IŠ VISKO, KAS BUVO PARAŠYTA AUKŠČIAU - ABIEJU BLOGŲ SCENARIJŲ BEI KANORS DAR SUGALVOT, NES 15 REIKIA. PAVYZDYS YRA 3.6 SKYRIUJE, TEN KAIP TIK APIE SKRYDŽIUS ŠNEKA



\section{Statistinių analizių vedėjų poreikiai}
Šiame skyriuje bus pateikiama detali statistinių analizių vedėjų poreikių, siekių ir kitų tikslų analizė.

\subsection{Einamųjų veiklų analizė}
Toliau bus pateiktos kelios einamosios veiklos, analizuojamai naudotojų grupei keliančios nepatogumus.

\subsubsection{Kompiuterizuojama veikla}
[TRUMPAS VIENO SAKINIO SCENARIJUAUS APRAŠYMAS]

\textit{TAS BLOGAS NAUDOJIMO SCENARIJUS, ŠIUO METU KELIANTIS STATISTIKAMS PROBLEMAS}

Pateiktame apraše išdėstytos esamos veiklos problemos ir neišnaudotos galimybės:  
\begin{itemize}
\item ČIA SURAŠOMOS VISOS TOS PROBLEMOS, KURIOS MATOMOS SCENARIJUJE
\end{itemize}

\subsection{Naudotojų ir veiklų charakteristikos}
DAR VIENAS BLOKAS TEKSTO PAGAL TAI, KAS YRA PARAŠYTA DABARTINĖJE MŪSŲ BIBLIJOJE 3.4 SKYRIUJE

\subsection{Būsimasis panaudojimo scenarijus}
Toliau bus pateiktas pavyzdinis panaudojimo scenarijus, kuris padės išspręsti statistinių analizių vedėjams kylančias problemas

KAIP KOMENTARUOSE SĄRAŠAS SU VISU SCENARIJUMI

\subsection{Naudotojo poreikia bei panaudojamumo siekiai}
VĖL PARAŠYTI DAR KELIS POREIKIUS PAGAL 3.6 IŠ BIBLIJOS

\section{Įkvėpiančios interfeisų idėjos}
SKYRIUS APIE INTERFEISUS

\printbibliography[heading=bibintoc, title=Šaltiniai]  % Šaltinių sąraše nurodoma panaudota
\end{document}


%Vartotojas su draugu nori skristi į mėnulį šios dienos pabaigoje:
%\begin{enumerate}
%\item  Vartotojas supildo informaciją apie skrydį.
%\begin{enumerate}[label*=\arabic*.]
%\item Pažymi, jog skris du žmonės.
%\item Pažymi, jog skris tik į vieną pusę.
%\item Pasirenka, jog skris iš Žemės.
%\item  Pasirenka, jog išvykti nori šiandien po 18:00 val.
%\end{enumerate}
%\item  Sistema pateikia sąrašą skrydžių, kurie vyksta iš Žemės ir kurių išvykimo laikas yra ne ankstesnis nei nurodyta vartotojo ir turi bent dvi laisvas vietas. %Prie kiekvieno skrydžio yra nurodyta kaina.
%\item Vartotojas pasirenką norimą skrydį.
%\item Sistema parodo to skrydžio išvykimo ir atvykimo datas su planuojamais laikais, skrydžio kainą bei galimas sėdimas vietas.
%\item Vartotojas pasirenka sėdimas vietas ir spaudžia mokėti.
%\begin{enumerate}[label*=\arabic*.]
%\item Vartotojas suveda asmens dokumento duomenis.
%\item Įveda elektroninio pašto adresą.
%\item Pasirenka apmokėjimo būdą.
%\item Susimoka už kelionę.
%\end{enumerate}
%\item Sumokėjus sistema išsiunčia  bilietus į el.paštą.
%\end{enumerate}

%Kelionių organizatorius nori suplanuoti klientų porai atostogas. Pirmiausia jie keliaus po Žemę, vėliau skris į Mėnulį ir atgal.
%\begin{enumerate}
%\item  Organizatorius supildo informaciją apie skrydį.
%\begin{enumerate}[label*=\arabic*.]
%\item Pažymi, jog skris du žmonės
%\item Pažymi, jog skris į abi puses.
%\item Pasirenka, jog skris iš Žemės.
%\item  Pasirenka, jog išvykti nori Spalio 13d., o grįžti Spalio 20d.
%\end{enumerate}
%\item  Sistema pateikia sąrašą skrydžių, kurie vyksta iš Žemės ir kurių išvykimo laikas yra ne ankstesnis nei nurodyta vartotojo, o išvykimo laikas ne %vėlesnis, ir turi bent dvi laisvas vietas. Prie kiekvieno skrydžio yra nurodyta kaina.
%\item Kelionių organizatorius pasirenka tokius skrydžius, kurie tinka jo kuriamam kelionės planui
%\item Sistema parodo tų skrydžių išvykimo ir atvykimo datas su planuojamais laikais, skrydžių kainą bei galimas sėdimas vietas.
%\item Organizatorius parodo planą klientams
%\item Organizatorius persiunčia bilietų nuorodas klientams
%\item Klientams duodama laiko pagalvoti, ar jiems patinka kelionės planas
%\end{enumerate}