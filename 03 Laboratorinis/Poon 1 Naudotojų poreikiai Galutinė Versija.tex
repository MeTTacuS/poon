\documentclass{VUMIFPSkursinis}
\usepackage{algorithmicx}
\usepackage{algorithm}
\usepackage{algpseudocode}
\usepackage{amsfonts}
\usepackage{amsmath}
\usepackage{bm}
\usepackage{caption}
\usepackage{color}
\usepackage{float}
\usepackage{graphicx}
\usepackage{listings}
\usepackage{subfig}
\usepackage{wrapfig}
\usepackage{enumitem}
\usepackage{multirow}
\usepackage[utf8]{inputenc}
%\usepackage{enumerate}

%PAKEISTA, tarpai tarp sąrašo elementų
\setitemize{noitemsep,topsep=0pt,parsep=0pt,partopsep=0pt}
\setenumerate{noitemsep,topsep=0pt,parsep=0pt,partopsep=0pt}

% Titulinio aprašas
\university{Vilniaus universitetas}
\faculty{Matematikos ir informatikos fakultetas}
\department{Programų sistemos}
\papertype{Laboratorinis darbas}
\title{Kelionių į Mėnulį maršrutų planavimo programa „Poon“ maketų euristinis tikrinimas}
\status{3 kurso 5 grupės studentai}
\author{Gabrielė Žielytė}
\secondauthor {Daumantas Šimkus}
\thirdauthor {Nedas Valentinovičius}
\titleineng{}

\date{Vilnius – \the\year, Galutinė versija}

% Nustatymai
\setmainfont{Palemonas}   % Pakeisti teksto šriftą į Palemonas (turi būti įdiegtas sistemoje)
\bibliography{bibliografija}

\begin{document}
	
% PAKEISTA	
\maketitle

%TURINYS
\thispagestyle{empty}
\tableofcontents


\sectionnonumnocontent{Anotacija}
Šiame dokumente aprašomas bilietų pirkimo kelionėms į mėnulį telefoninės programos „Poon“ panaudojamumo maketų euristinis tikrinimas. Kiekvieno komandos nario maketas įvertinamas kitų komandos narių, o jų radiniai pateikiami šiame dokumente. Studentų, dirbusių prie šio projekto, kontaktai:
\begin{itemize}
\item Gabrielė Žielytė - gabriele.zielyte@mif.stud.vu.lt. 
\item Daumantas Šimkus - daumantas.simkus@mif.stud.vu.lt. 
\item Nedas Valentinovičius - nedas.valentinovicius@mif.stud.vu.lt.
\end{itemize}
\thispagestyle{empty}

\cleardoublepage\pagenumbering{arabic}
\setcounter{page}{4}

\section{Įvadas}
\textbf{Programų sistemos pavadinimas: } Kelionių į Mėnulį maršrutų planavimo programa „Poon“
\bigskip

\textbf{Trumpasis pavadinimas: } „Poon“
\bigskip




\textbf{Projekto aprašas: } Kelionėms į Mėnulį tapus realybe, tapo ypač svarbu tvarkinga, aiški, moderni ir svarbiausia saugi kelionių planavimo sistema. Į Mėnulį plūsta žmonės iš skirtingų kultūrų, tikėjimo, puoselėjantys įvairias vertybes. Todėl Poon misija yra gerbti kiekvieną norintįjį skristi, parūpinti jam saugią ir malonią aplinką su kuo aiškesnėmis taisyklėmis kelionėms tarp Žemės ir Mėnulio.


\section{Gabrielės Žielytės maketo vertinimas}
\subsection{Daumanto Šimkaus įvertinimas}
\subsubsection{Euristinis tikrinimas}
\begin{center}
 \begin{tabular}{|| p{4cm} | p{4cm} | p{8cm} ||} 
 \hline
 Euristika & Defekto sunkumas & Komentaras \\
 \hline\hline
 Būsenos matomumas: grįžimas atgal & Didelis & Nusipirkus bilietus galima grįžti į nupirktų bilietų langą\\ 
 \hline
\end{tabular}	
\end{center}

\subsubsection{Sunkesnių pažeidimų apžvalga}
Sunkesnių pažeidimų nėra.

\subsubsection{Apibendrinimas}
Informacijos architektūra išdėstyta intuityviai.

kalba paprasta, pasirinkimai lengvai suprantami.

Naudojimo būdas nuoseklus ir lengvai perprantamas.

Dizainas  patrauklus, programa gali naudotis tiek neregys tiek matantysis.

\subsection{Nedo Valentinovičiaus įvertinimas}
\subsubsection{Euristinis tikrinimas}
\begin{center}
 \begin{tabular}{|| p{4cm} | p{4cm} | p{8cm} ||} 
 \hline
 Euristika & Defekto sunkumas & Komentaras \\
 \hline\hline
 Sistemos atititikimas realiam pasauliui & Vidutinis & Įvedant detales apie skrydį, negalima pasirinkti suaugusiųjų ir/ar vaikų kiekio, arba tai nėra aišku iš pirmo žvilgsnio. \\ 
 \hline
 Klaidų prevencija & Vidutinis & Pačioje bilietų pirkimo proceso pabaigoje vartotojui nėra parodytos jo užsakymo detalės, todėl nėra užkertamas kelias galimoms klaidoms \\
 \hline
\end{tabular}	
\end{center}

\subsubsection{Sunkesnių pažeidimų apžvalga}
Sunkesnių pažeidimų nebuvo.

\subsubsection{Apibendrinimas}
Šio maketo dizainas yra aiškus, neperkrautas nereikalinga informacija. Svarbiais momentais vartotojui leidžiama patvirtinti savo pasirinkimus „popup“ languose, taip sukuriant nemenką klaidu prevenciją.

Naudojama kalba yra aiški ir suprantama vartotojui, langai nėra perkrauti nereikalinga informacija, visi mygtukai per interaktyvūs langai yra tose vietose, kuriose tikimasi pažvelgus į langą pirmą kartą.

Pirkimo krovimosi langas rodo, kad vyskta informacijos apdorojimas, tačiau neturi aiškaus būdo parodyti, kiek laiko dar reikės laukti, kol bus galima patekti į kitą langą. Tai nekantresniems vartotojams gali sukelti problemų.

\section{Daumanto Šimkaus maketo vertinimas}
\subsection{Gabrielės Žielytės įvertinimas}
\subsubsection{Euristinis tikrinimas}
\begin{center}
 \begin{tabular}{|| p{4cm} | p{4cm} | p{8cm} ||} 
 \hline
 Euristika & Defekto sunkumas & Komentaras \\
 \hline\hline
 Būsenos matomumas & Sunkus & Nėra nei vieno „pop-up" lango, klausiančio vartotojo, ar šis yra tikras savo pasirinkimu.\\ 
 \hline
 Būsenos matomumas & Vidutinis & Vartotojui gali būti nelabai aišku, kad norint patekti į kitą langą, reikia braukti pirštu.\\
 \hline
 Būsenos matomumas: atsakas & Nedidelis & Nėra krovimo („loading") langų, kurie nurodytų vartotojui, jog sistema krauna langą. \\
 \hline
 Būsenos matomumas & Vidutinis & Ne iki galo parodoma informacijos architekrūra: lango viršuje nėra nurodyta, kur vartotojas tuo metu yra. \\  
 \hline   
 Naudotojo valdomas dialogas: aiškūs išėjimai & Vidutinis & Nėra mygtuko „quit" bilietų pirkimo operacijos pabaigoje arba ankstesniuose languose. \\
 \hline              
\end{tabular}	
\end{center}

\subsubsection{Sunkesnių pažeidimų apžvalga}
Sistemoje nėra nei vieno „pop-up" lango, klausiančio vartotojo, ar šis yra tikras savo pasirinkimu. Tokio lango egzistavimas yra reikalingas, pavyzdžiui po to, kai vartotojas pasirenka sėdėjimo vietas. Vartotojas gali būti netyčia pasirinkęs ne tas vietas, kurių norėjęs - „pop-up" langas paklaustų vartotojo, ar jis tikras, ir tuomet vartotojui norint būtų leista pasirinkti kitas vietas.

\subsubsection{Apibendrinimas}
Šiame makete vartotojas yra gana nuosekliai informuotas apie tai, ką jis yra pasirinkęs ir, kas vyksta dabartinėje būsenoje, bet trūksta „pop-up" ir krovimo langų, bei aiškumo, norint pereiti į kitą langą.

Naudojama suprantama kalba naudotojui, dialogas paprastas, vaizdas įprastas, atitinka naudotojo įpročius.

Sistema leidžia naudotojui grįžti į praeitą langą, yra rodomos nukeliautas kelias (apskritimai lango apačioje), bet nėra „quit" mygtuko.

Langų elementai išdėstyti darniai, bet ne stiliumi iš viršaus į apačią. Pasirinkimai aiškūs, išvaizda tokia pati visoje sistemoje.

Vartotojas yra apsaugotas nuo klaidingo pasirinkimo ar duomenų įvedimo, lango apačioje raudona spalva nurodant, kad pasirinkimas netinkamas. Meniu hierarchija paprasta, naudotojas nepasiklysta.

Dizainas nėra perkrautas, dėmesys kreipiamas į užduoties atlikimą.

\subsection{Nedo Valentinovičiaus įvertinimas}
\subsubsection{Euristinis tikrinimas}
\begin{center}
 \begin{tabular}{|| p{4cm} | p{4cm} | p{8cm} ||} 
 \hline
 Euristika & Defekto sunkumas & Komentaras \\
 \hline\hline
 Sistemos būsenos matomumas & Sunkus & Nėra nė vieno lango, parodančio vartotojui, kad vyksta informacijos ar duomenų procesavimas. \\ 
 \hline
 Naudotojo valdomas dialogas & Sunkus & Nėra nė vieno „popup“ lango, klausiančio vartotojo, ar jis yra įsitikęs savo sprendimu. \\ 
 \hline
 Estetiškas ir minimalistinis dizainas & Vidutinis & Didžioji programos dalis yra suskirstyta į gana didelius blokus, kurie gali nepatikti kai kuriems žmonėms. \\ 
 \hline
\end{tabular}	
\end{center}

\subsubsection{Sunkesnių pažeidimų apžvalga}
Langų, nerodančių duomenų ar informacijos apdorojimo proceso nebūvimas yra gana didelis trūkumas, nes vartotojas negali žinoti, ar programa dėl kokio nors defekto yra užstrigusi ar vyksta įprastas duomenų apdorojimas.

„Popup“ langų nebūvimas taip pat yra bėda, nes vartotojas nėra verčiamas įsitikinti, kad jo pateikti duomenys yra teisingi ir nebuvo atlikta kokia nors klaida.

\subsubsection{Apibendrinimas}
Šis maketas buvo sunkiai skaitomas, nes interaktyvūs mygtukai nebuvo paryškinti. 

Tam tikri sunkūs trūkimai gana smarkiai sumažina šio maketo panaudojamumą, tačiau unikali ir gana intuityvi vartotojo sąsaja padaro šį maketą įdomų. Programėle yra naviguojama ir iš kairės į dešinę, ir iį viršaus į apačią, tad visi telefonų teikiami privalumai yra gerai išnaudojami. 

Vartotojas yra apsaugotas nuo klaidino duomenų įvedimo pasinaudojus ne „popup“ langus, o kitą, skirtingų spalvų būdą. Tai yra unikali inonavija, tad naudotojams gali prireikti laiko priprasti prie jos, tačiau tai netrukdo visos sistemos panaudojamumui.

\section{Nedo Valentinovičiaus maketo vertinimas}
\subsection{Gabrielės Žielytės įvertinimas}
\subsubsection{Euristinis tikrinimas}
\begin{center}
 \begin{tabular}{|| p{4cm} | p{4cm} | p{8cm} ||} 
 \hline
 Euristika & Defekto sunkumas & Komentaras \\
 \hline\hline
 Būsenos matomumas: atsakas & Nedidelis & Sistema neparodo, kiek dar liko padaryti veiksmų iki galutinio rezultato.\\ 
 \hline
\end{tabular}	
\end{center}

\subsubsection{Sunkesnių pažeidimų apžvalga}
Sunkesnių pažeidimų nėra.

\subsubsection{Apibendrinimas}
Informacijos architektūra parodyta tinkamai, vartotojas visada įspėjamas apie tai, kas vyksta: yra „loading", bei „pop-up" langų.

Kalba yra suprantama naudotojui, nėra retesnių terminų, suprantamai pateikti pasirinkimai. Langų struktūra pastovi.

Naudotojas nėra pririštas prie vieno naudojimo būdo: yra galimybė atšaukti, grįžti, išeiti.

Klaidų prevencija: daugumoje pasirinkimo laukų yra pateikti naudingi standartiniai pasirinkimai, pvz.: laukelyje „Select file format" galima pasirinkti tik iš nurodytų variantų.

Dizainas nėra perkrautas, dėmesys kreipiamas į užduoties atlikimą.

\subsection{Daumanto Šimkaus įvertinimas}
\subsubsection{Euristinis tikrinimas}
\begin{center}
 \begin{tabular}{|| p{4cm} | p{4cm} | p{8cm} ||} 
 \hline
 Euristika & Defekto sunkumas & Komentaras \\
 \hline\hline
 Būsenos matomumas: sąrašo pasirinkimas & Nedidelis & Negalima pasirinkti visų galimų variantų sąrašuose\\ 
 \hline
 Būsenos matomumas: migtukas & Nedidelis & Atidarius sąrašą kompanijų, sąrašas uždengia grįžimo atgal mygtuką\\ 
 \hline
\end{tabular}	
\end{center}

\subsubsection{Sunkesnių pažeidimų apžvalga}
Sunkesnių pažeidimų nėra.

\subsubsection{Apibendrinimas}
Informacijos architektūra išdėstyta paprastai, yra „loading", bei „pop-up" langų.

Kalba yra be sudėtingesnių sąvokų, pasirinkimai yra intuityvūs ir kartu dar paaiškinti mygtukų pavadinimais. Langų struktūra pastovi.

Naudotojas nėra pririštas prie vieno naudojimo būdo: yra galimybė atšaukti, grįžti, išeiti.

Dizainas nėra patrauklus, bet labai paprastas, dėmesys kreipiamas į užduoties atlikimą ir vartotojo sugebėjimą orientuotis.

\section{Išvados}
Laboratorinio darbo išvadų skyriuje apibendrinami vertinimai ir pateikiamos detaliojo prototipo kūrimo gairės


















\printbibliography[heading=bibintoc, title=Šaltiniai]  % Šaltinių sąraše nurodoma panaudota
\end{document}
