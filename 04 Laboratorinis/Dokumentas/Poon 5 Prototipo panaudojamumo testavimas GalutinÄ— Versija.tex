\documentclass{VUMIFPSkursinis}
\usepackage{algorithmicx}
\usepackage{algorithm}
\usepackage{algpseudocode}
\usepackage{amsfonts}
\usepackage{amsmath}
\usepackage{bm}
\usepackage{caption}
\usepackage{color}
\usepackage{float}
\usepackage{graphicx}
\usepackage{listings}
\usepackage{subfig}
\usepackage{wrapfig}
\usepackage{enumitem}
\usepackage{multirow}
\usepackage[utf8]{inputenc}
%\usepackage{enumerate}

%PAKEISTA, tarpai tarp sąrašo elementų
\setitemize{noitemsep,topsep=0pt,parsep=0pt,partopsep=0pt}
\setenumerate{noitemsep,topsep=0pt,parsep=0pt,partopsep=0pt}

% Titulinio aprašas
\university{Vilniaus universitetas}
\faculty{Matematikos ir informatikos fakultetas}
\department{Programų sistemos}
\papertype{Laboratorinis darbas}
\title{Kelionių į Mėnulį maršrutų planavimo programos „Poon“ prototipo panaudojamumo testavimas}
\status{3 kurso 5 grupės studentai}
\author{Gabrielė Žielytė}
\secondauthor {Daumantas Šimkus}
\thirdauthor {Nedas Valentinovičius}
\titleineng{}

\date{Vilnius – \the\year, Galutinė versija}

% Nustatymai
\setmainfont{Palemonas}   % Pakeisti teksto šriftą į Palemonas (turi būti įdiegtas sistemoje)
\bibliography{bibliografija}

\begin{document}
	
% PAKEISTA	
\maketitle

%TURINYS
\thispagestyle{empty}
\tableofcontents

\sectionnonumnocontent{Anotacija}
Šiame dokumente aprašomas bilietų pirkimo kelionėms į mėnulį telefoninės programos „Poon“ prototipo panaudojamumo testavimas. Kiekvienas komandos narys prisidėjo prie prototipo kūrimo naudodami „JustInMind“ prototipavimo priemonę. Studentų, dirbusių prie šio projekto, kontaktai:
\begin{itemize}
\item Gabrielė Žielytė - gabriele.zielyte@mif.stud.vu.lt. 
\item Daumantas Šimkus - daumantas.simkus@mif.stud.vu.lt. 
\item Nedas Valentinovičius - nedas.valentinovicius@mif.stud.vu.lt.
\end{itemize}
\thispagestyle{empty}

\cleardoublepage\pagenumbering{arabic}
\setcounter{page}{4}

\section{Santrauka}
Šiame dokumente bus patekti „Poon“ prototipo testavimo rezultatai. Prototipą testavo kruopščiai atrinkti penki asmenys, kurie atitinka panaudojamomo scenarijus. Juos sudaro statistikos departamento darbuotojas, neregys, senjoras, žmogus perkantis kelionės bilietus tik sau bei žmogus, bilietus perkantis sau ir draugams. Testavimą vertino „Poon“ trijų kūrėjų komanda, testuotojams davusi prieeigą prie programos ir prižiūrėjusi visą testavimo procesą.

\section{Įvadas}
Kelionių į Mėnulį bilietų pirkimo programos „Poon“ prototipas kurtas pagal apibrėžtus maketus pasinaudojant nemokama prototipavimo priemone „JustInMind“. Ši programa turi integruota apmokymo sistemą, kuri greitai ir nesudėtingai leido kurti prototipą, atsižvelgiant į vartotojų poreikius.


\section{Testavimo aprašas}
\subsection{Testuojamos užduotys}
\begin{itemize}
\item Pirkti bilietą(us) \\
	\textbf{Kriterijai:}
	\begin{itemize}
	\item Sistema neišmetė lango, jog ne visi laukai užpildyti
	\item Pasirinktas skrydis tik į vieną pusę arba į abi
	\item Pasirinktas išvykimo miestas
	\item Pasirinktas atvykimo miestas
	\item Pasirinktos išvykimo atvykimo datos
	\item Pasirinktas teisingas keleivių skaičius
	\item Rastas patraukliausias skrydis(-džiai)
	\item Pasirinta teisingai keleivių lytis
	\item Suvesti keleivių vardai ir pavardės
	\item Pasirinktos sėdėjimo vietos
	\item Paspaustas mygtukas „Tvirtinti“ duomenis
	\item Neprireikė vertintojo pagalbos
	\item Atlikta greičiau nei per 2min
	\end{itemize}
	\textbf{Sekmės matas:} Pasiektas bilietų apmokėjimo langas \\

\item Sumokėti už kelionės bilietą(us)  \\
	\textbf{Kriterijai:}
	\begin{itemize}
	\item Sistema neišmetė lango, jog ne visi laukai užpildyti
	\item Neprireikė vertintojo pagalbos
	\item Pasirinktas mokėjimo būdas
	\end{itemize}
	\textbf{Sekmės matas:} Sistema išmetė langą, jog bilietai sėkmingai nupirkti \\

\item Parsisiųsti statistiką  \\
	\textbf{Kriterijai:}
	\begin{itemize}
	\item Nunaviguota į statistikos langą
	\item Pasirinkta kompanija
	\item Pasirinktas duomenų formatas
	\item Neprireikė vertintojo pagalbos
	\item Užduotis atlikta greičiau nei per 1min
	\end{itemize}
	\textbf{Sekmės matas:} Sistema išmetė langą, jog statistika sėkmingai parsisiųsta \\

\item Pakeisti nustatymus  \\
	\textbf{Kriterijai:}
	\begin{itemize}
	\item Nunaviguota į nustatymų langą
	\item Pakeistas nustatymas
	\item Neprireikė vertintojo pagalbos
	\item Atlikta per mažiau nei minutę
	\end{itemize}
	\textbf{Sekmės matas:} Pakeitus nustatymus sėkmingai grįžta į pradinį langą \\
\end{itemize}

\subsection{Metodas}
Testavimui buvo pasirinktas metodas kai vertintojas viso testavimo proceso metu stebi testuotoją ir seka jo progresą, o iškilus bėdai suteikia pagalbos pasižymėdamas, jog testavimas neįvyko nepriekaištingai. Testuotojas buvo supažindintas su užduotimis kurias jis turi atlikti.

\subsection{Aplinka}
Testavimas vyksta testuotojui natūralioje aplinkoje t. y. jo namuose, darbe ar miesto kavinėje. Testuotojui buvo duotas išmanusis telefonas su iOS operacine sistema, kurioje yra įrašyta  „Poon“ prototipavimo programa.

\subsection{Dalyviai}

dalyvių charakteristikų lentelė, sukurta remiantis klausimynų atsakymais.

\section{Testavimo rezultatai}
\subsubsection{Užduočių vykdymo rezultatai}
dalyvių rezultatai sugrupuojami pagal užduotis.
	\begin{itemize}
	\item 1-oji užduotis
	\item 2-oji užduotis
	\item 3-oji užduotis
	\item 4-oji užduotis
	\end{itemize}
\subsubsection{Dalyvių komentarai}
komentarai taip pat grupuojami pagal užduotis
	\begin{itemize}
	\item 1-oji užduotis
	\item 2-oji užduotis
	\item 3-oji užduotis
	\item 4-oji užduotis
	\end{itemize}

\section{Rekomendacijos}
Rekomendacijų skyriuje analizuojamas rastų defektų sunkumas ir dažnis, skaičiuojami prioritetai, pateikiami siūlomidefektų sprendimai.

\section{Priedai}
\subsection{Klausimynai}
\subsection{Dalyvių rezultatų lentelės}
\subsection{Dalyvio sutikimo dalyvauti testavime raštas}





1. paprastas zmogus, perkantis sau
2. statistikos departamento darbuotojas
3. neregys ( like bbz kaip tai padaryt )
4. uhhh zmogus, perkantis keliems zmonems
5. diedukas, kuris vos moka telefonu naudotis













\printbibliography[heading=bibintoc, title=Šaltiniai]  % Šaltinių sąraše nurodoma panaudota
\end{document}
